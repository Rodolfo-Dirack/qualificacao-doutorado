\chapter{INTRODUÇÃO}
\label{cap1}

A equação do sobretempo normal \cite{dix} é uma aproximação de tempo de trânsito de reflexão válida para pequenos afastamentos
entre os pares fonte receptor na superfície de registro em uma aquisição sísmica. Esta equação foi desenvolvida para modelos de
multicamadas planas horizontais. 
A equação de sobretempo normal é uma aproximação em série de Taylor de segunda ordem para o tempo de trânsito, por isto é
também chamada de aproximação hiperbólica. Todavia, esta aproximação diverge do tempo de trânsito analítico
em grandes afastamentos.

Aproximações de tempo de trânsito não hiperbólicas foram desenvolvidas na literatura,
com o intuito de estender a região de convergência das
aproximações de tempo de trânsito no domínio do afastamento: Estas aproximações utilizam mais de dois termos para aumentar a 
acurácia da análise de velocidades e a correção de sobretempo normal. Cada uma destas aproximações terá
as suas limitações na estimativa das velocidades e na relação afastamento profundidade.

O empilhamento convencional, utilizando as aproximações de sobretempo normal hiperbólicas ou não hiperbólicas, é estendido
para o domínio do Ponto Médio Comun (PMC) a partir do empilhamento Superfície de Reflexão Comum (SRC): 
O empilhamento é realizado sobre uma superfície de tempo de
trânsito no domínio do meio afastamento $h$ e na vizinhança de um PMC central $m_0$, a partir de
três parâmetros ($R_N$, $R_{NIP}$ e $\beta_0$) dados para cada $m_0$.
O empilhamento SRC possui uma aproximação de tempo de trânsito hiperbólica \cite{jager}, e várias aproximações
de tempo de trânsito SRC foram propostas com o objetivo de estender a região de convergência desta aproximação:
As aproximações do SRC não hiperbólico \cite{fomel1}, SRC quarta ordem
\cite{germam}.

Um caso especial do método de empilhamento SRC, é o empilhamento por Elemento de Reflexão Comum (ERC):
O empilhamento ERC é também realizado no domínio do afastamento $h$ e na vizinhança de um PMC central $m_0$, assim como o
empilhamento SRC. Porém, o empilhamento ERC é feito sobre uma curva de tempo de trânsito de reflexão correspondente ao
conjunto de trajetórias de reflexão que possuem em comun
o mesmo ponto de incidência sobre o refletor.
Este método possui a vantagem de prover
parâmetros importantes para a construção do modelo de velocidades, e utiliza dois dos parâmetros
do método SRC ($R_{NIP}$ e $\beta_0$).
A principal desvantagem do método ERC é a interpolação: O empilhamento ERC necessita de 
traços em coordenadas de PMC $m$ e meio afastamento $h$ não coincidentes com as coordenadas
e amostragem dos dados adquiridos.
O objetivo desta pesquisa é propor uma metodologia de inversão do modelo de velocidades utilizando o método ERC para
a obtenção da seção empilhada. 

Propomos a seguinte estratégia de interpolação para a obtenção da seção empilhada ERC:
O algoritmo Very Fast Simulated Aneeling (VFSA) é utilizado para obter os parâmetros do SRC de afastamentos nulo
$R_{NIP}$ e $\beta_0$. Como a obtenção de tais parâmetros é um subproduto do empilhamento SRC esta etapa é comum
aos dois métodos, possibilitanto que os dois sejam realizados em fluxos de trabalho independentes que compartilham os 
mesmos parâmetros, não sendo necessário nenhum procedimento além do usual para o empilhamento SRC convencional.
A trajetória ERC é traçada no plano $m, h$ para estabelecer as coordenadas dos traços pertencentes à família ERC
com uma equação que descreve a trajetória e o a equação de tempo de trânsito ERC \cite{cre}.
Interpolaremos os dados adiquiridos utilizando os filtros adaptativos de predição de erro (FPE) no domínio PMC \cite{liu11}.
Esta etapa possibilita a amostragem adequada para a obtenção das famílias ERC,
uma para cada PMC central $m_0$. 
Finalizadas estas etapas,
obteremos a seção empilhada ERC a partir do empilhamento das amostras sobre uma curva de tempo de trânsito ERC 
pertencentes as famílias ERC. Esta curva é determinada para cadar par $m_0, t_0$ da seção empilhada \cite{cre}.

%% Refazer este parágrafo com a estratégia de otimização do modelo de velocidades
A principal vantagem do método ERC é não precisar de informação a priori do macromodelo de velocidades. 
Este método poder ser utilizado na obtenção do modelo de velocidades através de alguma etapa de inversão,
como a tomografia \cite{cre}. Para a obtenção do modelo de velocidades propomos a seguinte estratégia:
Utilização do algoritmo de offset continuation de modo a obter um painel de velocidades para cada ponto sobre a seção
empilhada, este modelo será otimizado de maneira iterativa através da repetição deste processo.







