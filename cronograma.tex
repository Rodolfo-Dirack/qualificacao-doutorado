\chapter{CRONOGRAMA}
\label{cap9:cronograma}


\section{Etapas concluídas}

A seguir a descrição das etapas concluídas e o cronograma:

  \begin{enumerate}
   \item  Testes preliminares do algoritmos, Pesquisa e fundamentação teórica (09/2017 - 08/2019): Produção de testes 
   de implementação
   dos algoritmos utilizados e primeiras versões dos algoritmos originais do Autor.
   Pesquisa das principais referências teóricas e estabelecimento do tema central da tese.
   \item   Modelagem Kirchhoff (09/2019): Utilização do algoritmo de modelagem 
   do pacote Madagascar\footnote{Madagascar é um pacote
   de procesamento sísmico 'open source' disponível em \url{http://www.ahay.org/wiki/Main_Page}.}
   \textit{sfkirmod} para produzir os dados do modelo do refletor
   gaussiano.
    \item Obtenção dos parâmetros do SRC utilizando o VFSA (09/2019): Programa \textit{sfvfsacrenh} escrito  pelo Autor
    em linguagem C e adaptado para o pacote Madagascar
   baseado no algoritmo Very Fast Simulated Aneeling \cite{ingber}. O algoritmo ajusta a superfície de tempo de trânsito
   do SRC não hiperbólico (Equação \ref{eq:2.4}) aos dados modelados. Os parâmetros do SRC que produzem o melhor ajuste são
   os parâmetros otimizados.
    \item  Interpolação do cubo de dados com FPE (09/2019): Interpolação das seções de
    afastamento constante extraídas dos dados modelados
    (chamado ``cubo de dados'') a partir de Filtros Adaptativos de Predição de Erro
    com os programas \textit{sfapef} e \textit{sfmiss4} do pacote Madagascar.
    Esta interpolação permite a discretização
    suficiente dos traços no domínio do PMC para possibilitar a correta amostragem das famílias ERC.
    \item  Cálculo das trajetórias ERC (09/2019): Programa \textit{sfcretrajec} desenvolvido pelo Autor em linguagem C 
    e adaptado para o pacote Madagascar para
    o cálculo das trajetorias ERC baseado na Equação \ref{eq:2.1}.
     \item Obtenção das famílias ERC (09/2019): Programa \textit{sfgetcregather} desenvolvido pelo Autor em linguagem C 
     e adaptado para o pacote Madagascar para a
     determinação dos traços sísmicos do cubo de dados que estão sobre as trajetorias ERC previamente calculadas na etapa anterior.
     Estes traços formam as famílias ERC.
    \item  Cálculo das curvas de empilhamento ERC (10/2019): Programa \textit{sfgetcretimecurve} desenvolvido pelo Autor 
    em linguagem C e adaptado 
    para o pacote Madagascar para a determinação das curvas de tempo de trânsito ERC com auxílio das 
    Equações \ref{eq:2.3}-\ref{eq:2.4}.
     \item Paralelização do algoritmo com scons (12/2019): Utilização de técnicas de computação paralela para 
     melhorar o desempenho
     dos algoritmos desenvolvidos. O pacote Madagascar permite a Paralelização dos processos realizados a partir da execução
     com o comando \textit{scons -j\#} onde '\#' representa o número de núcleos utilizados.
    \item  Empilhamento e seção empilhada ERC (12/2019): Programa \textit{sfcrestack} 
    desenvolvido pelo Autor em linguagem C e adaptado 
    para o pacote Madagascar para a obtenção da seção  empilhada ERC.
  \end{enumerate}

    \begin{table}[H]
      \caption{Cronograma de trabalho das etapas já concluídas em cada mês do ano de 2019.}
      \centering
      
      \begin{tabular}{|p{6cm}|c|c|c|c|c|}%{|c|c|c|c|c|c|}

     \hline
      \textbf{Etapas concluídas} & 08/19 & 09/19 & 10/19 & 11/19 & 12/19 \\ \hline
      Pesquisa e fundamentação teórica & x & & &  & \\ \hline
      Modelagem Kirchhoff & & x & &  & \\ \hline
      Obtenção dos parâmetros do SRC utilizando o VFSA & & x & &  & \\ \hline
      Interpolação do cubo de dados com FPE & & x & &  & \\ \hline
      Cálculo das trajetórias ERC & & x & &  & \\ \hline
      Obtenção das famílias ERC & & x & &  & \\ \hline
      Cálculo das curvas de empilhamento ERC & & & x & & \\ \hline
      Paralelização do algoritmo com scons & & & & & x \\ \hline
      Empilhamento e seção empilhada ERC & & & &  & x \\
      \hline
      
      \end{tabular}
  \end{table}
  
A seguir as etapas concluídas após a qualificação de doutorado e até a entrega deste relatório:
  
  \begin{enumerate}
  \item Qualificação da tese (03/2020): Defesa e apresentação deste relatório de qualificação de tese.
  \item Desenvolvimento da estratégia de inversão do modelo de velocidades através da simulação
  de difrações \cite{diffractions}, esta estratégia foi abandonada, pois necessita do
  modelo de velocidades de sobretempo normal para simular as difrações (2020).
  \item Pesquisa teórica sobre a estratégia de inversão baseada na NIP tomografia \cite{niptomo}
  e stereotomografia \cite{stereo} e protótipos dos programas para o traçamento de raios (2020).
   \item Programas sfnipmodsetup e sfgetparameter desenvolvidos pelo Autor em linguagem C 
     e adaptados para o pacote Madagascar para a configuração inicial das fontes pontuais PIN
    e obtenção dos parâmetros $R_{NIP}$ e $\beta_0$ (03/2021).
   \item Desenvolvimento do programa sfzgradtomo desenvolvido pelo Autor em linguagem C 
    e adaptado para o pacote Madagascar para a inversão do modelo de velocidades de background
    (velocidade variando linearmente com a profundidade) (04/2021).
   \item Desenvolvimento do protótipo do programa sfstereoniptomo desenvolvido pelo Autor em linguagem C 
     e adaptado para o pacote Madagascar para a inversão do modelo de velocidades com variação
     lateral de velocidades (06/2021)
  \end{enumerate}


  \begin{table}[H]
      \caption{Cronograma de trabalho das etapas já concluídas em 2020 e 2021.}
      \centering
      
      \begin{tabular}{|p{6cm}|c|c|c|c|c|}%{|c|c|c|c|c|c|}

     \hline
      \textbf{Etapas concluídas} & 2020 & 03/21 & 04/21 & 05/21 & 06/21 \\ \hline
      Qualificação de doutorado & x & & &  & \\ \hline
      Inversão com simulação de difrações & x & & &  & \\ \hline
      Inversão baseada na NIP e stereo tomografia & x & & &  & \\ \hline
      Programas configuração inicial do modelo & & x & &  & \\ \hline
      Inversão modelo gradiente constante & & & x & x & \\ \hline
      Inversão variação lateral de velocidade (protótipo) & & & & & x \\
      \hline
      \end{tabular}
  \end{table}
  
\section{Etapas ainda não concluídas}

A seguir a descrição das atividades ainda não realizadas e o cronograma:

  \begin{enumerate}
    \item Inversão do modelo de velocidades com variação lateral (01/2022): Desenvolver o algoritmo de
    inversão do programa sfstereoniptomo completo.
    \item Testar a estratégia em outros modelos (02/2022): A estratégia de inversão será testada em 
    vários modelos de velocidades diferentes, em camadas com variações lateral de velocidade dentro da
    camada.
    \item  Escrita do texto da tese (08/2022): Escrever e revisar o texto da tese, apresentar os resultados da
    inversão do modelo de velocidades.
     \item Apresentação ao comitê de avaliação (08/2022): A data da apresentação para o Comitê de Avaliação de Tese é o dia 01 de Agosto de 2022.
    \item  Defesa de tese (08/2022): A data da apresentação para a Banca Avaliadora dependerá das sugestões do
    Comitê de Avaliação de Tese. A Previsão para defesa é entre os meses de Agosto de
    2022 e Setembro de 2022 (Já considerando a prorrogação do prazo de defesa).
  \end{enumerate}
  
   \begin{table}[H]
      \caption{Cronograma de trabalho até o prazo final da defesa de tese.}
      \centering
      
      \begin{tabular}{|p{6cm}|c|c|c|c|c|}

      \hline
      \textbf{Etapas ainda não concluídas} & 01/21 & 02/22 & 03/22 & 08/22 \\ \hline
      Inversão do modelo de velocidades & x & x & & \\ \hline
      Testes em vários modelos & & & x & \\ \hline
      Escrita do texto da tese & & & & x \\ \hline
      Apresentação ao comitê de avaliação & & & & x \\ \hline
      Defesa de tese & & & & x \\
      \hline
      
      \end{tabular}
  \end{table}
  
