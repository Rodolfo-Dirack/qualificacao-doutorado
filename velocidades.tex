\chapter{INVERSÃO DO MODELO DE VELOCIDADES}
\label{cap8:velocidades}

Propomos a seguinte metodologia para a inversão do modelo de velocidades e cumprimento do objetivo da tese:
Utilizar a simulação de resposta de difrações com a aproximação de tempo de trânsito ERC (Equação \ref{eq:2.3}) ou
a aproximação de tempo de trânsito do SRC não hiperbólico (Equação \ref{eq:2.4}), 
com aplicação da condição SDC ($R_N=R_{NIP}$),
para simular as difrações \cite{diffractions}.

O conceito de determinação da velocidade através da simulação de resposta de difração
consiste dos seguintes passos \cite{diffractions}:

\begin{enumerate}
 \item Determinação da velocidade NMO: Isto pode ser feito em cascata nos pontos da
seção empilhada, camada por camada otimizando a velocidade uma camada por vez.
\item Determinação dos slopes de reflexão de afastamento nulo no cubo de dados empilhados: 
Este passo não será necessário na nova metodologia.
\item Transformação dos traços zero offset selecionados em respostas de pontos difratores em CDP's
selecionados.
\item Aplicação da migração em profundidade pós empilhamento com análise de velocidades
em looping.
\end{enumerate}

Utilizando cada CMP da malha do modelo como um ponto espalhador de energia (Ponto difrator), propomos 
simular uma fonte pontual explosiva e obter a seção empilhada com várias hipérboles de difração. 
Daí utilizaremos a metodologia do artigo Post-stack velocity analysis by separation and imaging of seismic 
diffractions para a obtenção das velocidades de migração.